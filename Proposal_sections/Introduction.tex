% If you want to compile this section only, make sure to include relevant document headers and the \being \end document commands.
% You can make this a bit easier if you use the subfile package

Urban Green Space (UGS) is a fundamental part of the urban environment that generates economic benefits \parencite{kim_economic_2018}, supports urban residents’ physical and mental health \parencite{russo_modern_2018}, and serves as a city landmark that constitutes the city image \parencite{groos_millennium_2008}. UGS has received significant academic attention in recent years due to the COVID-19 pandemic, which made the public realize the importance of physical and mental well-being. Many studies have been initiated and conducted to understand if and how interacting with UGS can alleviate stress and enhance mental wellness during the pandemic lockdown. However, previous research projects usually relied on governmental data portals to collect UGS distribution data, which neglects minor green space and private green space, causing potential data inadequacy. I hypothesize that such a data collection methodology might underestimate the value of UGS. Without a comprehensive and objective understanding of the UGS distribution, it is challenging to study how different urban residents utilize and interact with UGS differently. Economic segregation might lead to serious mental disadvantages, making the urban environment less friendly to economically disadvantaged groups. 

That is why I am interested in understanding the functionalities and benefits of UGS as an economic booster for the local real estate market, a recreational facility that provides mental relief, and a city landmark that attracts global visitors. For this research proposal, I plan to first demonstrate the importance and significance of UGS research in social science disciplines, including geography, psychology, and urban studies, by reviewing previous studies that analyzed how UGS can shape personal experience and influence people’s mental wellness based on the study of spatial cognition. Then I will visit different methodologies that utilize the hedonic regression model to understand the relationship between locational features and property price. Based on the theoretical framework and well-established real estate valuation methodology, I will explore the current knowledge gap caused by the data deficiency, based on which, I will propose my research plan and enhanced data methodology.


