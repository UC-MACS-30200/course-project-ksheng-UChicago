% If you want to compile this section only, make sure to include relevant document headers and the \being \end document commands.
According to contemporary urban planners, the minimum UGS amount is 9 square meters per capita, and the ideal UGS amount is 50 square meters per capita \parencite{russo_modern_2018}. As an essential component of the urban environment, UGS contributes to the urban economy, provides stress alleviation, and constitutes the city's image. In this literature review, I will first analyze research projects related to the major benefits of UGS and then approach UGS with the spatial cognition framework to understand the significance of UGS in geography, psychology, and urban studies. Finally, I will conclude with a literature review of the traditional methodology of UGS evaluation and the potential data deficiency of the data collection process to initiate my research design and propose my data enhancement methodology. 

Previous researchers focused on understanding the benefits of UGS. Researchers studied Chicago’s Millennium Park and found that the economic benefits provided by land appreciation and business booming have greatly exceeded the investment and management of the park \parencite{groos_millennium_2008}. Researchers also claimed a positive correlation between the interaction with the natural landscape and residents’ physical and mental well-being \parencite{bratman_impacts_2012}. This correlation has been recognized across different times, different geographic regions, and different academic disciplines, including Public Health, Environmental Psychology, and Urban Planning \parencite{wendelboe-nelson_scoping_2019}. 

A study conducted in Australia also found that greener spaces are contributing to both mental and physical health among adults in middle-to-older age \parencite{astell-burt_mental_2013}. Locally, a study conducted at the University of Michigan to test and compare individual attention restoration capabilities in natural vs. urban environments demonstrates that the natural environment is more peaceful and enjoyable than any other environment. Long exposure to the natural scene or frequent interaction with the natural environment can enhance one’s directed attention abilities \parencite{berman_cognitive_2008}. A later study investigating the association between access to urban green space and mental health found that UGS proximity and the proportion of UGS in a larger community are usually associated with decreasing anxiety \parencite{nutsford_ecological_2013}. Based on this finding, more recent research studied the impact of COVID-19 on Canadian’s physical activity behaviors and found that outdoor physical activities can effectively alleviate anxiety \parencite{lesser_impact_2020}. A Chicago-based research team also supported this claim by conducting comparative research in two indoor spaces: Garfield Park Conservatory (UGS) and Water Tower Place Mall (non-UGS). They found that UGS is more associated with positive feelings and creativity \parencite{schertz_neighborhood_2021}. In conclusion, UGS is not only important but also essential in human’s daily functioning \parencite{bertram_role_2015}.

UGS is also significant in city image building. In many global cities, UGS provides a symbolic image as a city landmark. In different social contexts, the landmark can carry different socio-cultural meanings. Central Park has been widely accepted as a New York City landmark for decades \parencite{rosenzweig_park_1992}. Chicago’s Millennium Park was designed to be a symbolic icon that attracts tourists and booms the local economy \parencite{groos_millennium_2008}. In Japan’s capital city, Tokyo, Hibiya Park is located right in front of the Imperial Palace, which is also close to the National Diet Building. The park’s proximity to the political centers makes itself an ideal location for urban rioters, who seized upon the open spaces to rise against and overturn the top-down rule \parencite{gordon_crowd_1988}.

All the different functionalities and benefits related to UGS can be understood in the context of spatial cognition. Geographers and psychologists have tried organizing these characteristics using Lynch’s spatial cognition framework. Lynch outlined five elements of the community cognitive map in The Image of the City: paths, edges, nodes, districts, and landmarks. These elements constitute the methodology by which people understand the urban/community environment around them \parencite{lynch_image_1960}. Based on this framework, studies have been conducted to examine the functionality and spatial cognition of UGS. 

A study conducted in Beijing examined the different types of UGS in the city through cognitive mapping \parencite{hou_residents_2021}. The result showed that all five elements are presented in Beijing’s UGS. What is more, different types of UGS have differences in the distribution of the respondents’ cognitive maps. The differences are due to the inherent form of UGS. Landmark UGS serves as a beacon to the outside world by delivering prominent information that is easily perceived by visitors. It usually incorporates significant landscape structures and rich cultural carrying contents. What is more, a recently published paper further supported the spatial cognition of UGS by summarizing the three major components constituting one’s natural experience: natural interactions, circumstances, and internal responses \parencite{danrakedzon_framework_2024}. Researchers found that interacting with the outer environment can trigger internal responses under specific circumstances. 

Previous researches show that UGS is significant in economic growth, mental health, and landmark construction. UGS also plays an important role in Lynch’s spatial cognition framework. However, is it possible to quantify the value of UGS objectively? Researchers studying housing prices give us some ideas. In past decades, studies utilized hedonic regression model to understand the economic effects of UGS on residential real estate. Nicholls and Crompton, who studied the housing market near greenways in Austin, Texas, found a 20\% value premium in properties abutting green spaces \parencite{nicholls_impact_2005}. Kim and Peiser found a 25.5\% value premium in properties with a view of large, passive recreational greenways in the planned community of Los Angeles, CA \parencite{kim_economic_2018}. In the early 2000s, the city of Chicago launched a billion-dollar investment in implementing UGS including the Millennium Park and the Palmisano Nature Park. The successful investment made Chicago one of the world’s leading cities in UGS implementation. A study of Chicago utilizing the hedonic regression model found that larger park has a positive influence on the local property price \parencite{shaikh_economic_2011}, while Chen et al. further extended the traditional hedonic model by incorporating the gravity model and found a positive relation between property price and the accessibility of green spaces \parencite{chen_spatial_2022}. 

Even though the results were promising, there was a significant data deficiency in their research designs. The data methodology of previous researches only focused on government-planned parks and green spaces. The government's official database only included the planned UGS but ignored private green space and informal/small-scale green space in the city. Without considering a comprehensive and objective analysis of UGS distribution would lead to an underestimated evaluation of UGS \parencite{feltynowski_challenges_2018}, making UGS accessibility unequal among different social groups, and exacerbating urban segregation. In the city, private backyards, sidewalk trees, and lawns also play an important role in defining how green the neighborhood is, so in my research design and data collection, I plan to enhance the green space distribution data by incorporating satellite imagery and geographic information system (GIS) technology. Meanwhile, by leveraging web scrapping tools, I have collected the most updated housing prices and related housing characteristics as control factors.

% You can make this a bit easier if you use the subfile package