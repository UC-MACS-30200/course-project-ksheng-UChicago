% If you want to compile this section only, make sure to include relevant document headers and the \being \end document commands.
\documentclass{article}
\usepackage[utf8]{inputenc}

\title{Literature Review on Urban Green Space}
\author{Kuang Sheng}
\date{April 22nd, 2024}

\begin{document}

\maketitle

Urban Green Space (UGS) has been an important part of the urban environment and shows continuous influence on urban residents’ mental health, urban social equality, and economic benefits in previous studies. Studying the actual value of UGS in the urban environment can help justify people’s natural preference for green space in the city, facilitate policies to alleviate urban segregation, and create significant economic benefits by improving land value.

UGS has been proved to have a correlation with residents’ mental and psychological health. On a global level, Russo and Cirella stated that there is a worldwide trend of urban green space (UGS) enhancing urban well-being and livelihood by listing a series of UGS utilization examples in compacted urban spaces like Singapore, Dubai, and Hong Kong \cite{russo_modern_2018}. On a local level, an Australian-based research concluded that there is a significant relationship between urban green space and urban residents’ mental health based on the finding that middle-to-older aged adults, with a lower risk of psychological issues, tend to adopt an active lifestyle in greener spaces \cite{astell-burt_mental_2013}. When researchers explore this idea in North America, they pointed out that in large urban areas, the natural environment varies between neighborhoods causing neighborhood segregation. Meanwhile, the natural environment provides psychological functioning with significant benefits (Berman, 2021).

However, not all UGS serves the same functionality and provides the same level of benefit. A study found that the variables of walking distance to large scale, passive green space are usually associated with increasing housing prices, while those of walking distance to golf courses and active recreational space are usually negatively associated with housing prices \cite{kim_economic_2018}. Green space types are more significantly related to housing price than size, as types bring more variations in coefficient values for recreational and private green spaces \cite{chen_spatial_2022}. Eshtiyagh et al. further utilized a machine learning model on different neighborhood features to predict the chance of gentrification and found that green space, as a geographic community feature, can effectively predict the possibility of gentrification \cite{eshtiyagh_graph-based_2023}. Since gentrification is a major contributor and cause of rising housing prices in the United States, in which underprivileged residents are more vulnerable, understanding the value of different types of UGS can help alleviate and limit growing segregation in American cities.

Historically, the hedonic regression model has been widely used to estimate housing prices, which reflects the economic value of property. Nicholls and Crompton found a 20\% value premium in properties abutting green spaces in a study conducted in Austin, TX \cite{nicholls_impact_2005}. A Chicago-based study found that green spaces nearer to neighborhoods positively influence the housing market regarding both walking and driving accessibilities \cite{chen_spatial_2022}.

However, previous researchers rely on government data for UGS distribution data, which can be insufficient, inadequate, and biased. The government official database only included the planned UGS and ignored private green space and informal/small-scale green space in the city \cite{feltynowski_challenges_2018}. This problem can be resolved by incorporating satellite imagery data and deep learning methodologies. Researchers in Mexico conducted deep learning identification on UGS in Mexican metropolitan areas \cite{huerta2021}. By providing high-resolution satellite imagery, the deep learning models processed and studied the NDVI value and identified different types of UGS in the city. Such technology and data advancement can help enhance the quality of UGS distribution data and examine the true value of UGS in major global cities.

\bibliographystyle{plain}
\bibliography{references0321.bib}

\end{document}

% You can make this a bit easier if you use the subfile package