\documentclass{article}

\title{Big Picture Draft}
\author{Kuang Sheng}
\date{April 22, 2024}

\begin{document}


\maketitle
 

\section*{Proposed Title}
What is the true value of Urban Green Space (UGS) using satellite imagery data?

\section{Introduction \& Literature Review}

\subsection*{Context}
In previous studies, social scientists, urban planners, and psychologists have revealed the positive benefits of implementing Urban Green Space (UGS) in cities around the world. However, the current UGS distribution data seems to be insufficient due to governmental data collection limitations in most studies. An objective evaluation of UGS using satellite imagery is necessary for comprehensive analysis.


\subsection*{Your Research Question}
What is the true value of Urban Green Space (UGS) in a major metropolitan area? I plan to use Chicago as an example and compare the governmental UGS data source with the NDVI result based on satellite imagery analysis. Using a machine learning model to test if the governmental UGS data are inadequate in UGS analysis and if the enhanced data collection method can enhance machine learning performance. I also plan to conduct additional MTurk research on the subjective evaluation of UGS.

\subsection*{What does the existing literature say}
The existing literature summarizes studies conducted in cities around the world, revealing the benefits of UGS towards urban residents' well-being and mental health. However, other literature points out the deficiency of governmental UGS data and proposes incorporating satellite imagery to enrich the research dataset.

\subsection*{Significance with respect to existing knowledge}
Researchers seldom pay close attention to the accuracy of governmental data, which can be biased. City government data tend to focus on their achievement, instead of showing objective status quo. Adopting more comprehensive data sources can enhance data accuracy and research quality.

\section{Data and Methods}
\subsection*{State data and justify}
I plan to download Landsat satellite scenes from the United States Geological Survey (USGS) and calculate the normalized difference vegetation index (NDVI). Then, I will use a hedonic/linear regression machine learning model to test the different results by incorporating governmental green space data vs. satellite imagery green space data to compare the difference. Finally, I will collect subjective evaluation of UGS from MTurk by survey design.


\subsection*{State method and justify}
Analytical methods include NDVI calculation, hedonic/linear regression, and MTurk data analysis. A pilot study has applied a linear regression machine learning model to some property-related features, which shows adequate feasibility. NDVI calculation has been widely used in remote sensing and agricultural analysis. I am less familiar with MTurk, so I will start exploring more on the data collection and data analysis of this part.


\section{Feasibility}
\subsection*{Evaluation of approach w.r.t. RQ/project goal}
This study can contribute to evaluating the true benefits/value of UGS since the current result might show underestimation due to insufficient data sources. My research design can open new doors for UGS analysis and other urban studies relying solely on governmental data collection.

\subsection*{Initial Results}
A pilot study shows a negative correlation between public parks (governmental data) and housing value in Chicago, which does not match with previous studies. I assume this is due to the insufficient governmental data.

\subsection*{Proposed Timeline}
Data Collection: Ready for the Redfin part, 2 weeks for USGS Satellite Imagery; NDVI calculation: 3 weeks, Machine Learning Model Adjustment: 3 weeks, MTurk Survey Design and Data Collection: 3 weeks

\subsection*{Securing an Advisor}
Ali Sanaei/Zhao Wang - MACSS and Yue Lin/Crystal Bae - Center for Spatial Data Science

\subsection*{Cost}
Costs of MTurk Usage

\section*{Assessment of the overall structure and alignment}
This project is suitable for my educational background and work experience. This study can contribute to evaluating the true benefits/value of UGS since the current result might show underestimation due to insufficient data sources. My research design can open new doors for UGS analysis and other urban studies relying solely on governmental data collection.

\end{document}