\documentclass{article}

\usepackage{hyperref}
\usepackage[style = apa]{biblatex} % Adjust "style" as you see fit
   \addbibresource{references0321.bib}  % replace with your own references.bib when you create it
   
\usepackage{graphicx} % Used to display examples. You can remove it from the final document
\usepackage{placeins}

\usepackage{geometry}
    \geometry{margin=1in} %setting smaller margins to better display examples

\title{Annotated Bibliography}
\author{Kuang Sheng}
\date{March 23, 2024}

\begin{document}

\maketitle


\textbf{Link to my Github Repo:} \href{Add link to your github repo here}{https://github.com/UC-MACS-30200/course-project-ksheng-UChicago}\\
\textbf{Research Topic/Question:} how much does urban green space (UGS) cost in Chicago’s residential real estate?


\begin{enumerate}
    \item  \fullcite{astell-burt_mental_2013}\\ \\ This paper summarizes a study conducted in New South Wales, Australia between 2006 and 2009, based on one’s psychological distress and the percentage green space around one’s residence. This study conducted a multilevel logit regression and concluded that for middle-to-older age adults, more green space has a significantly positive influence on people’s psychological health if they adopt an active lifestyle. This paper demonstrates the importance of green space implementation by building and illustrating a strong relationship between urban green space and urban resident’s mental health. The study depicts green space as a positive factor in determining residential quality and provides my research questions with adequate theoretical support in relating urban green space to property values and urban social justice.
    
    \item  \fullcite{russo_modern_2018}\\ \\ This paper further illustrates the worldwide trend of urban green space (UGS) enhancing urban well-being and livelihood. The paper listed a series of UGS utilization examples in compacted urban spaces like Singapore, Dubai, and Hong Kong, to show the diversity of UGS distribution. UGS is not limited to well-planned park space. Rooftop green space, and sidewalk trees can also be counted as UGS. This paper helped identify the existence and the influence of minor UGS, which was usually neglected by the methodology of previous research designs on the topic of UGS.

    \item  \fullcite{nicholls_impact_2005}\\ \\ In past decades, studies utilized hedonic regression to understand the economic effects of UGS on residential real estate. Nicholls and Crompton, who studied the housing market near greenways in Austin, Texas, found a 20 percent value premium in properties abutting green spaces. This paper provides a solid description and evidence for analyzing housing prices using a hedonic regression model. In my research design, the hedonic regression model will be applied to web-scrapped housing prices and local green space percentages based on satellite imagery.

    \item  \fullcite{chen_spatial_2022}\\ \\ This paper explored the relation between the housing price and its connectivity and accessibility to green space in Chicago. This research methodology went beyond the traditional research design, which only considered real estate’s proximity to green space. By incorporating time-based accessibility of green space, this research confirmed that green space has a positive influence on the housing price based on both walking and driving distance. It also confirmed that the size of the green space also influences the housing price. However, only planned parks and major green spaces are included in the analysis. My research methodology will enrich the data by incorporating minor/informal green space in Chicago.

    \item  \fullcite{kim_economic_2018}\\ \\ This paper explores and analyzes the economic effects of different types of green spaces in different neighborhoods. This research combined hedonic regression and GIS analysis to quantitatively analyze the economic benefits of UGS. The study confirmed that the greatest economic benefits occur in the planned neighborhood with passive green space. The study provided a unique perspective by including different types of UGS and categorizing them based on the type of neighborhood they are located. For my research, I expect to see different pricing patterns in compacted city areas (like Chicago Loop) and in the fringe of the city, analysis by categorization in this research design can provide my research design with framework reference.

    \item  \fullcite{huerta_mapping_2021}\\ \\ This paper recorded two deep-learning methodologies for identifying UGS in Mexican metropolitan areas. By providing high-resolution satellite imagery, the deep learning models processed and studied the NDVI value and identified different types of UGS in the city. The methodology can be incorporated into my research by providing possible deep-learning tools and models for identifying and quantifying UGS near the listed residential properties.

    \item  \fullcite{feltynowski_challenges_2018}\\ \\ This article compares the data of UGS among different data sources and finds significant disparities between them. The government's official database only included the planned UGS and ignored private green space and informal/small-scale green space in the city. I emphasize the importance of objectively evaluating the UGS distribution in the city within my research design. This article serves as theoretical support for my proposal to include satellite imagery and crowdsourcing geospatial data in my UGS analysis.

     \item  \fullcite{eshtiyagh_graph-based_2023}\\ \\ This research incorporates machine learning methodologies to study the different factors influencing gentrification in major US cities. One of the factors considered in the machine learning tasks was green space identification based on satellite imagery. Gentrification is one of the major causes of housing price appreciation in the US. Understanding the relationship between UGS and gentrification can help explain the segregation between urban neighborhoods and evaluate the economic value of UGS.
\end{enumerate}


\end{document}

